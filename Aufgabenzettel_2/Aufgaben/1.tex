\section*{\nr.1 \titone (25 Punkte)}
\begin{enumerate}[(a)]
\item Mit $U=RI$ und $I=Q/t$ ergibt sich eine erzeugte Ladung von 
\begin{equation}
  Q=\frac{U}{2Rt}
\end{equation}
mit der Bedingung, dass 
\begin{equation}
  Q=n_1e\cdot0.25\cdot10^4
\end{equation}
sein muss (wobei $n_1$ die auf den Photomultiplier getroffenen Elektronen und e die Elementarladung bezeichnen) ergibt sich
\begin{equation}
   n_1=\frac{Ut}{2Re\cdot 0.25\cdot10^4}=2.5\cdot10^{12}.
 \end{equation} 
Der Photomultiplier mit Durchmesser $d$ bedeckt nur ein kleines Raumelement $A$ der gesamten Kugeloberfläche $O$ im Abstand $a$, weshalb sich für die Gesamtzahl $n_0$ der emittierten Photonen
\begin{equation}
   n_0=n_1 \frac{O}{A}= 16 n_1 \left(\frac{a}{d}\right)^2=2.5\cdot10^{18}
 \end{equation} 
 ergibt.
 \item Auf die Pupille des Mannes im Mond mit Durchmesser $d=0.6$cm fallen in der Entfernung von $a=384400$km noch 
 \begin{equation}
   n_2=n_0 \frac{A}{O}=n_0 \left(\frac{d}{4a}\right)^2=3.8\cdot 10^{-6}
  \label{eq:1}
 \end{equation}
 Photonen, es fällt also nur mit einer sehr kleinen Wahrscheinlichkeit überhaupt ein Photon in die Pupille des Mannes im Mond. \emph{Anmerkung:} Uns ist nicht genau klar, wie die Pupille des Mannes im Mond zu interpretieren ist und welche Entfernungen wirklich gemeint sind, weshalb wir mit oben definierten Größen gerechnet haben.
 \item Mit \vref{eq:1} ergibt sich für einen Spionagesatelliten im Abstand $a=300$km mit einem Durchmesser von $2.5$m $n_3=1.09\cdot10^7$.

\end{enumerate}
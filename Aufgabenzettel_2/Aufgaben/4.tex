\section*{\nr.4 \titfour (25 Punkte)}
\begin{enumerate}[(a)]
\item Die spektrale Energiedichte der Planck-Strahlung ist durch
\begin{equation}
u(\nu)  = \frac{8\pi h\nu^3}{c^3}\frac{1}{\exp\left[h\nu/( k T)\right]-1 }
\end{equation}
gegeben.
Die spektrale Energiedichte ist über $u(\nu) = h \nu n(\nu)$ mit der spektralen Anzahldichte der Photonen verknüpft, was aus Energiebilanz in einem Volumen $\mathrm{d}V$ folgt:
\begin{equation}
u(\nu)\mathrm{d}\nu \mathrm{d}V = h \nu n(\nu) \mathrm{d}\nu \mathrm{d}V
\end{equation}
Es folgt:
\begin{equation}
n(\nu)  = \frac{8\pi\nu^2}{c^3}\frac{1}{\exp\left[h\nu/( k T)\right]-1 }
\end{equation}
Die Grenzfrequenz $\nu_\text{min}$ berechnet sich über $\epsilon= h\nu_\text{min}$ zu $\nu_\text{min} = \SI{1.09e15}{\hertz}$.
\begin{figure}[htbp]
\centering
\input{anzahldichte.tex}
\caption{Anzahldichte der Photonen der Sonne.}
\label{fig:anzahldichte}
\end{figure}
\item Der Anteil $q$ derjeniger Photonen, die in der Lage sind, Elektronen durch den Photoeffekt herauszulösen, ist das Verhältnis der Anzahldichte der Photonen überhalb der Grenzfrequenz und der Gesamtanzahldichte:
\begin{equation}
q = \frac{\int_{\nu_\text{min}}^{\infty} n(\nu) \, \mathrm{d}\nu}{\int_{0}^{\infty} n(\nu) \, \mathrm{d}\nu}
\end{equation}
Numerische Integration mit Python liefert für den Anteil $q\approx \num{0.0052}$.
\end{enumerate}
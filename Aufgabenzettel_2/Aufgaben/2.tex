\section*{\nr.2 \tittwo (25 Punkte)}
\begin{enumerate}[(a)]
\item Es gilt
\begin{equation}
  \Delta \lambda=\lambda_s-\lambda_0 =2\lambda_c\sin^2\left(\frac{\varphi}{2}\right)
\end{equation}
Mit der Bedingung 
\begin{equation}
  \frac{\mathrm{d}\Delta\lambda}{\mathrm{d}\varphi}=\lambda_c\sin(\varphi)=0
\end{equation}
folgt, dass die größte Wellenlängenänderung für $\varphi=180^{\circ}$ erfolgt, diese ist dann $\Delta \lambda_{max}=2\lambda_c$.
Für große Wellenlängen ist diese Änderung nur schwer nachweisbar, da für $\lambda_c\ll\lambda$ die relative Änderung der Wellenlänge nur sehr klein ist.
\item Für die Compton-Streuung gilt für $\theta=120^{\circ}$
\begin{equation}
  \lambda'=\lambda_0+\frac{3}{2}\lambda_c
\end{equation}
mit $\lambda_0=\frac{ch}{E}$ folgt
\begin{equation}
  \lambda'=\frac{ch}{E}+\frac{3}{2}\lambda_c
\end{equation}
und mit $p'=\frac{h}{\lambda'}$ folgt
\begin{equation}
  p'=\frac{h}{\frac{ch}{E}+\frac{3}{2}\lambda_c}
\end{equation}
Die kinetische Energie des Elektrons ergibt sich aus der Energieänderung des Photons
\begin{equation}
  E_{kin}=E_0-E'=E_0- \frac{ch}{\lambda'}=0.00075E_0
\end{equation}
\item Für den Impuls des Photons gilt
\begin{equation}
  p_{ph}=\frac{E_{ph}}{c}.
\end{equation}
Aus der relativistischen Energie-Impuls-Beziehung ergibt sich
\begin{equation}
  p_{el}=\frac{\sqrt{E^2-E_0^2}}{c}
\end{equation}
mit $E_{ph}=E-E_0$ und $E_0=E_{ph}$ folgt daraus
\begin{equation}
  p=\frac{\sqrt{E_{ph}^2+2E_0E_{ph}}}{c}=\frac{\sqrt{3}}{c}E_{ph}
\end{equation}
Da dieser Impuls größer ist als der Impuls des einfallenden Photons muss das gesamte Kathodenmaterial einen Impuls in entgegengesetzter Richtung erhalten um die Impulserhaltung nicht zu verletzen.
\end{enumerate}
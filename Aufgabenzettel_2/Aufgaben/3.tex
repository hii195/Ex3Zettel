\section*{\nr.3 \titthree (25 Punkte)}
\begin{enumerate}[(a)]
\item Sein $N_\gamma$ die Anzahl von Photonen mit Frequenz $\nu$, die in einem bestimmten Zeitraum auf die Solarzelle treffen. Diese bringen eine Energie von
\begin{equation}
E = h \nu N_\gamma
\end{equation}
mit. Durch die Absorption durch die Solarzelle werden statistisch $\eta N_\gamma$ Elektronen aus ihrem Gitter gelöst, wobei $\eta$ die Quanteneffizienz bezeichnet. Die betragsmäßige Ladung eines Elektrons beträgt eine Elementarladung $e$, sodass die Gesamtladung
\begin{equation}
Q = \eta N_\gamma e
\end{equation} 
im betrachteten Zeitpunkt herausgeschlagen wird.
Für die Responsivität folgt mit $c=\lambda \nu$:
\begin{equation}
R = \frac{Q}{E} = \frac{\eta N_\gamma e}{h \nu N_\gamma} = \frac{\eta e}{h \nu} = \frac{\eta e}{hc} \lambda
\end{equation}
So ergibt sich der in der Abbildung zu erklärende proportionale Zusammenhang zwischen Responsivität und Wellenlänge. Solarzellen sind effizient gebaut, sodass uns vernünftig erschien, die Quanteneffizienz in den folgenden Rechnungen auf $\eta = 1$ zu setzen. So ergibt sich für die Steigung des idealisierten Verlaufs ein Wert von ca. \SI{8e5}{\ampere\per\watt\per\meter}, der sich mit dem Wert deckt, den man durch graphisches Ablesen der Steigung erhält.
\item Eine Skizze der spektralen Strahlungsleistung der Sonne pro Fläche ist durch \vref{fig:sonnenspektrum} gegeben.
\begin{figure}[htbp]
\centering
\input{sonnenspektrum.tex}
\caption{Spektrale Strahlungsleistung der Sonne.}
\label{fig:sonnenspektrum}
\end{figure}

\item Sei $u(\lambda)$ die spektrale Energiedichte nach Planck. Licht eines Wellenlängenbereichs $\mathrm{d}\lambda$ strahlt im Abstand $d$ von der Sonne mit Radius $R_s$ mit einer Strahlungsleistung $\mathrm{d}S$ pro Fläche von
\begin{equation}
\mathrm{d}S = \frac{c}{4} u(\lambda) \mathrm{d}\lambda \cdot \frac{4\pi R_s^2}{4\pi d^2},
\end{equation}
wobei der Vorfaktor $c/4$ aus dem Lambertschen Gesetz folgt und auf dem letzten Zettel hergeleitet wurde. Dieses Strahlungsdifferential führt zu einem kleinen Strom von:
\begin{equation}
\mathrm{d} I_{pv} = A R \mathrm{d}S = \frac{2\pi c \eta e A \,\mathrm{d}\lambda}{\lambda^4\left\{\exp\left[hc/(\lambda k T)\right]-1 \right\}} \cdot \frac{R_s^2}{d^2}
\end{equation}
Der Gesamtstrom ergibt sich durch Integration über einen Wellenlängenbereich $\left[\lambda_\text{min}, \lambda_\text{max} \right]$:
\begin{equation}
I_{pv} = \int_{\lambda_\text{min}}^{\lambda_\text{max}}\mathrm{d} I_{pv}
\end{equation}
Aus der Abbildung ergibt sich ein vernünftiger Wellenlängenbereich von $\left[\SI{400}{\nano\meter}, \SI{1000}{\nano\meter} \right]$, in dem die Responsivität ideal ist. Außerhalb ist sie näherungsweise null. Numerische Integration mit Python liefert $I_{pv}=\SI{4.38}{\ampere}$.  
\end{enumerate}
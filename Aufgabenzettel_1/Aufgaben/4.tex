\section*{\nr.4 \titfour (25 Punkte)}
\begin{enumerate}[(a)]
\item 
Die Sonne strahlt in alle Raumrichtungen insgesamt eine Leistung von
\begin{equation}
S' = 4\pi R_s^2\sigma T_s^4
\end{equation}
aus, wobei wobei $R_s$ und $T_s$ Sonnenradius und Sonnenoberflächentemperatur bezeichnen. Vom Planeten wird der Anteil absorbiert, der sich aus dem Raumwinkelanteil des Planetenquerschnitts mit Radius $R_p$ in der Sphärenfläche mit Radius $d$, dem Abstand zwischen Sonne und Planet, ergibt. Zusätzlich muss die Albedo $\alpha = \num{0.3}$ berücksichtigt werden. Insgesamt absorbiert der Planet die Leistung:
\begin{equation}
S = S'(1-\alpha) \cdot \frac{\pi R_p^2}{4\pi d^2} = \frac{\pi R_s^2\sigma T_s^4 R_p^2(1-\alpha)}{d^2}
\end{equation}
Der Planet selbst emittiert die Leistung
\begin{equation}
P = 4\pi R_p^2\sigma T_p^4,
\end{equation}
wobei $T_p$ die Planetenoberflächentemperatur bezeichnet. Gleichsetzen von absorbierter und emittierter Leistung liefert für den Abstand $d$:
\begin{equation}
d = \frac{R_s T_s^2 \sqrt{1-\alpha}}{T_p^2} \approx \SI{2.61e11}{\meter} = \SI{1.74}{\astronomicalunit}
\end{equation}
Dabei wurden $R_s = \SI{6.963e8}{\meter}$ und $T_s = \SI{5778}{\kelvin}$ recherchiert. Die Albedo entspricht näherungsweise die der Erde. Da die Erdtemperatur etwas über \SI{0}{\celsius} liegt, ist der oben errechnete Wert für den Abstand realistisch, auch wenn atmosphärische Effekte komplett vernachlässigt wurden (falls der Planet wie die Erde eine Atmosphäre besitzt).
\item 
Die Wellenlänge beim Maximum des Spektrums lässt sich über das Wiensche Verschiebungsgesetz errechnen:
\begin{equation}
\lambda_\text{max}= \frac{\SI{0.29}{\centi\meter\kelvin}}{\SI{5800}{\kelvin}}  = \SI{500}{\nano\meter}
\end{equation}
Das Maximum liegt also im sichtbaren Bereich des elektromagnetischen Spektrums.
\item
Führt man eine Messung an einem schwarzen Körper durch, so lässt sich die spektrale Energiedichte nicht direkt messen, sondern stattdessen wird die spektrale Strahlungsdichte $E$ gemessen, die proportional zur Energiedichte $u$ ist:
\begin{equation}
E \propto u
\end{equation}
Die Proportionalitätskonstante hängt vom Abstand des Messinstruments vom schwarzen Körper ab, ist jedoch für beide Messungen gleich, da bei gleichem Abstand gemessen wird. Zu zwei verschiedenen Wellenlängen $\lambda_{1,2}$ werden also die spektralen Strahlungsleistungen $E_{1,2}$ gemessen, sodass gilt:
\begin{align}
E_1 &= \frac{\mu}{\lambda_1^5\left\{ \exp[hc/(\lambda_1 k T)]-1\right\}} \\
E_2 &= \frac{\mu}{\lambda_2^5\left\{ \exp[hc/(\lambda_2 k T)]-1\right\}}
\end{align}
Dabei sind die Proportionalitätskonstante $\mu$ sowie die Temperatur $T$ des schwarzen Körpers unbekannt. Da die Exponentialfunktion bijektiv ist, wird dieses Gleichungssystem von zwei Gleichungen und zwei Unbekannten eindeutig gelöst, insbesondere lässt sich prinzipiell die Temperatur $T$ ermitteln. Allerdings lässt sich kein analytischer Ausdruck für die Temperatur finden, man muss sich also mit Näherungsverfahren begnügen.



\end{enumerate}
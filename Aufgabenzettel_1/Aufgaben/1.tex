\section*{\nr.1 \titone (25 Punkte)}
\begin{enumerate}[(a)]
\item Aus 
\begin{equation}
  u(\nu)=\frac{8\pi h\nu^3}{c^3\left(\exp\left(\frac{h\nu}{k_BT}\right)-1\right)}
\end{equation}
folgt mit $\nu=\frac{c}{\lambda}$, $u(\nu)\mathrm{d}\nu=u(\lambda)\mathrm{d}\lambda$ und $\frac{|\mathrm{d}\nu|}{|\mathrm{d}\lambda|}=\frac{c}{\lambda^2}$, 
\begin{equation}
  u(\lambda)=\frac{8\pi hc}{\lambda^5\left(\exp\left(\frac{hc}{k_BT\lambda}\right)-1\right)}
\end{equation}
\item Es gilt $c=\lambda\nu$ und, nach Einstein, $\mathrm{d}c=0$, woaus folgt, dass
\begin{equation}
  0=\mathrm{d}c=\mathrm{d}(\lambda\nu)=\lambda \mathrm{d}\nu+\nu \mathrm{d}\lambda.
\end{equation}
Damit ergibt sich
\begin{equation}
  \frac{\mathrm{d}\nu}{\nu}=-\frac{\mathrm{d}\lambda}{\lambda}.
\end{equation}

Die Verläufe $u(\nu)$ und $u(\lambda)$ sind in \vref{fig:spektren1} und \vref{fig:spektren2} dargestellt, die Abhängigkeit des Spektrums vom Planckschen Wirkungsquantum findet sich in \vref{fig:spektren3}.
Aus \ref{fig:spektren3} erkennt man, dass die Wahl vom Planckschen Wirkungsquantum einen sehr signifikanten Einfluss auf das Spektrum eines schwarzen Strahlers hat. Die erforderliche Genauigkeit des Werts für $h$ ist also sehr hoch.

\begin{figure}[htbp]
\centering
\input{spektren1}
\caption{Spektren von verschieden heißen schwarzen Körpern}
\label{fig:spektren1}
\end{figure}

\begin{figure}[htbp]
\centering
\input{spektren2} 
\caption{Spektren von verschieden heißen schwarzen Körpern}
\label{fig:spektren2}
\end{figure}

\begin{figure}[htbp]
\centering
\input{spektren3}
\caption{Spektren für verschiedene Werte von $h$}
\label{fig:spektren3}
\end{figure}

\item Es gilt 
\begin{equation}
  \exp(x)=\sum_{n=0}^{\infty}\frac{x^n}{n!}=1+x+\frac{x^2}{2}+\dots.
\end{equation}
Für $x\to0$ folgt daraus $\exp(x)-1=x$, da die Terme höherer Ordnung vernachlässigbar werden.
Damit folgt für große $\lambda$
\begin{equation}
  \frac{8\pi hc}{\lambda^5\left(\exp\left(\frac{hc}{k_BT\lambda}\right)-1\right)}\approx\frac{8\pi hc}{\lambda^5\frac{hc}{k_BT\lambda}}=\frac{8\pi k_BT}{\lambda^4}
\end{equation}
was das Rayleigh-Jeans-Gesetz ist.


\end{enumerate}
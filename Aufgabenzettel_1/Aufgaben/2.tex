\section*{\nr.2 \tittwo (25 Punkte)}
\begin{enumerate}[(a)]
\item Nähert man einen $1.75\mathrm{m}$ großen Menschen als Zylinder an, erhält man mit einem geschätzten Bauchumfang von $80\mathrm{cm}$ eine Oberfläche von etwa $1.5m^2$, da außerdem noch extra Hautfläche durch Abweichungen von diesem vereinfachten Modell existieren, wird die Gesamthautfläche auf $A=2m^2$ geschätzt. Eine solche Oberfläche strahlt eine Leistung von 
\begin{equation}
  P=\sigma AT^4=1049.3W
\end{equation}
ab.
\item Für die minimale Frequenz und damit die maximale Wellenlänge muss gelten
\begin{equation}
\frac{h\nu_{min}}{\exp(\frac{h\nu_{min}}{k_BT})}=h\nu_{min}
\end{equation}
woraus sich mit $\nu=\frac{c}{\lambda} $
\begin{equation}
\lambda_{max}=\frac{hc}{\ln(2)k_BT}=6.69\cdot10^{-5}m
\end{equation}
ergibt, die abgegebene Strahlugn liegt damit im Infraroten und ist deshalb nicht mit dem menschlichen Auge sichtbar.
\item Es gilt nach dem Stefan-Boltzmanngesetz
\begin{equation}
  P_{Netto}=(T_{Raum}^4-T_{Koerper}^4)\sigma A=-211.8W
\end{equation}

\end{enumerate}
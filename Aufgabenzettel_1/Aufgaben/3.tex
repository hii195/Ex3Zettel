\section*{\nr.3 \titthree (25 Punkte)}
\begin{enumerate}[(a)]
\item die spektrale Energiedichte ist durch
\begin{equation}
u(\lambda) = \frac{8\pi h c}{\lambda^5[\exp(hc/(\lambda k T))-1]}
\end{equation}
gegeben. Eine infinitesimale Änderung $\mathrm{d}F$ der pro Flächeninhalt abgestrahlten Leistung $F$ errechnet sich durch:
\begin{equation}
\mathrm{d}F = u(\lambda) \mathrm{d}\lambda \cdot c \cdot \cos{\theta} \cdot \frac{\mathrm{d}\Omega}{4\pi}
\end{equation}
Der Lichtgeschwindigkeitsfaktor $c$ macht aus der Energiedichte eine Strahlungsleistung pro Fläche, mit $\cos{\theta}$ wird berücksichtigt, dass nach dem Lambert'schen Gesetz nur über den Strahlungsanteil senkrecht zur aktuellen Oberfläche zu integrieren ist, und der Faktor $\mathrm{d}\Omega/(4\pi)$ quantifiziert den Raumwinkelanteil. Es folgt:
\begin{align}
F &=\int_{0}^{\infty}u(\lambda) \mathrm{d}\lambda \int_\text{Halbraum}c \cdot \cos{\theta} \cdot \frac{\mathrm{d}\Omega}{4\pi} \\
&=8\pi h c\int_{0}^{\infty} \frac{\mathrm{d}\lambda }{\lambda^5[\exp(hc/(\lambda k T))-1]}\int_{0}^{2\pi}\int_{0}^{\pi/2}c \cdot \cos{\theta} \cdot \frac{\sin{\theta}\mathrm{d}\phi\mathrm{d}\theta}{4\pi} \\
\intertext{Sei $y:=hc/(\lambda k T)$ und damit $\mathrm{d}\lambda=-\lambda^2 k T/(hc)\mathrm{d}y$:}
&= 4\pi hc^2 \left(\frac{kT}{hc} \right)^6\int_{0}^{\infty}y^5\frac{\mathrm{d}y}{\exp(y)-1} \left(\frac{hc}{kT} \right)^2 \frac{1}{y^2} \int_{0}^{\pi/2}\sin{\theta}\cos{\theta} \mathrm{d}\theta\\
\intertext{Sei $z:=\sin{\theta}$ und damit $\mathrm{d}z=\cos{\theta}\mathrm{d}\theta$:}
&= 4 \pi h c^2\left(\frac{kT}{hc} \right)^4 \int_{0}^{\infty}y^3\frac{\mathrm{d}y}{\exp(y)-1} \int_0^{1} z \mathrm{d}z \\
&= \frac{2\pi^5  k^4}{15h^3c^2} T^4 =: \sigma T^4
\end{align}
Mit $k = \SI{1.381e-23}{\joule\per\kelvin}$, $h = \SI{6.626e-34}{\joule\second}$ und $c=\SI{2.998e8}{\meter\per\second}$ folgt:
\begin{equation}
\sigma = \SI{5.670e-8}{\watt\per\meter\squared\per\kelvin\tothe{4}}
\end{equation}
Dieser Wert deckt sich mit dem Literaturwert aus der Vorlesung.

\item Statt $u(\lambda)$ abzuleiten und gleich null zu setzen, empfiehlt es sich, eine bijektive Funktion nachzuschalten, die die Rechnung vereinfacht. Die Lage der Maxima bleibt unverändert. Sei also
\begin{equation}
v(\lambda) := \ln\left(\frac{u(\lambda)}{8\pi hc} \right) = -5\ln{\lambda} - \ln\left[ \exp(hc/(\lambda k T))-1\right]
\end{equation}
Nach $\lambda$ Ableiten und gleich null Setzen liefert:
\begin{align}
0 &= -\frac{5}{\lambda} + \frac{\exp(hc/(\lambda k T))hc/(\lambda^2 k T)}{\exp(hc/(\lambda k T))-1} \\
\implies 0 &= -\frac{5kT\lambda}{hc} +\frac{\exp(hc/(\lambda k T))}{\exp(hc/(\lambda k T))-1} \\
\intertext{Die Substitution $y:=hc/(\lambda k T)$ und geschicktes Erweitern liefern:}
0 &= -\frac{5}{y} + \frac{1}{1-\exp(-y)} \\
\implies 0 &= 5\left[ 1-\exp(y)\right] - y 
\end{align}
Um diese nichtlineare Gleichung numerisch zu lösen, wurde ein Python-Skript geschrieben, das die Nullstelle der Funktion $f(y):=5\left[ 1-\exp(y)\right] - y$ approximiert, wobei das Programm bei einem einstellbar kleinen Fehler abbricht. Die Iterationsvorschrift lautet:
\begin{equation}
y_{n+1} = y_n - \frac{f(y_n)}{f'(y_n)}
\end{equation}

\lstinputlisting{newton_script.py}

Die ausgegebene Lösung lautet $y = \num{4.965}$. Mit einer Rücksubstitution folgt direkt:
\begin{equation}
\lambda_\text{max} T = \frac{hc}{yk} \approx \SI{2.897e-3}{\meter\kelvin}
\end{equation}
Dies ist bereits das Wiensche Verschiebungsgesetz.
\end{enumerate}
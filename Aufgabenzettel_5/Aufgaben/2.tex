\section*{\nr.2 \tittwo (25 Punkte)}
\begin{enumerate}[(a)]
\item Die $x$-Komponente des Impulses der in Richtung Objektiv gestreuten Photonen darf maximal eine Umbestimmtheit von
\begin{equation}
  \Delta p_x= p_{ph}\sin(\alpha)
\end{equation}
haben, wobei $p_ph=\frac{h}{\lambda}$ den Photonenimpuls und $\alpha$ den Winkel zwischen der y-Achse und der maximalen Ablenkung des Photons bezwichnet. Dabei gilt $\sin(\alpha)=\frac{d}{2y}$ woraus
\begin{equation}
  \Delta p_x=\frac{h}{\lambda}\frac{d}{2y}
\end{equation}
folgt.
\item Aus dem Rayleigh-Kriterium folgt, dass mit einem Mikroskop nicht genauer als 
\begin{equation}
  D=\frac{2y\lambda}{d}
\end{equation}
auflösen kann. Dies entspricht also der Maximalen Auflösung des Ortes des Elektrons $\Delta x=D$.
\item Es folgt
\begin{equation}
  \Delta x \Delta p = h
\end{equation}
Das Ergebnis ist also unabhängig vond er Wellenlänge des Lichts. Verwendet man kurzwelligeres Licht, so kann man zwar die Ortsauflösung erhöhen, die Impulsunschärfe wird jedoch dadurch erhöht und andersherum. 
\end{enumerate}

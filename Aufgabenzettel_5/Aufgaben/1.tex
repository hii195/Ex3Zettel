\section*{\nr.1 \titone (25 Punkte)}
\begin{enumerate}[(a)]
\item Befindet sich ein Teilchen der Masse $m$ in einem rechteckigen Potentialtopf mit Breite $w$ und Tiefe $d$, so ist die Unschärfe des Ortes mindestens durch die Breite des Potentials gegeben: $\Delta x = w$. Durch die Heisenbergsche Unschärferelation ist somit der Impuls $p$ durch $\Delta p = \hbar/(2w)$ beschränkt, wodurch die Mindestenergie
\begin{equation}
E_0 = \frac{p^2}{2m} \geq \frac{\hbar^2/(4w^2)}{2m} = \frac{h^2}{16\pi m w^2} > 0
\end{equation}
folgt.
\item Ist das Phasenraumvolumen durch $V = wd$ gegeben, und nimmt jeder Zustand eine Größe $h$ ein, so passen
\begin{equation}
n = \frac{wd}{h}
\end{equation}
Zustände in den Topf.
\end{enumerate}
\section*{\nr.2 \tittwo (25 Punkte)}
\begin{enumerate}[(a)]
\item Die zu untersuchende Wellenfunktion ist durch
\begin{equation}
\psi(x) = \left(\frac{m\omega}{\pi\hbar}\right)^{1/4} \exp\left(-\frac{m\omega}{2\hbar}x^2 \right)
\end{equation}
gegeben, was eine Gaußkurve um $x=0$ beschreibt. Auch das Betragsquadrat ist eine Gaußkurve mit dieser Eigenschaft, wodurch die Aufenthaltswahrscheinlichkeit für $x=0$ am größten ist.
\item
Transformieren wir zunächst die Wellenfunktion in den Impulsraum:
\begin{align}
\psi(p) &= \frac{1}{\sqrt{2\pi\hbar}}\int_{-\infty}^{\infty} {\phi(x)\exp\left(-i\frac{px}{\hbar} \right)\,\mathrm{d}x}\\
 &= \left(\frac{1}{m\omega\pi\hbar}\right)^{1/4} \exp\left(-\frac{1}{2m\omega\hbar}p^2 \right)
\end{align}
Eine weitere Zutat ist:
\begin{equation}
\langle p^2\rangle = \int_{-\infty}^{\infty} {p^2\left|\psi(p)\right|^2\,\mathrm{d}p} 
= \frac{1}{2}m\omega\hbar
\end{equation}
Es folgt für die kinetische Energie:
\begin{equation}
\langle T \rangle = \frac{\langle p^2\rangle}{2m} = \frac{1}{4} \omega \hbar
\end{equation}
Weiter berechnet man:
\begin{equation}
\langle x^2\rangle = \int_{-\infty}^{\infty} {x^2\left|\psi(x)\right|^2\,\mathrm{d}x} 
= \frac{\hbar}{2m\omega}
\end{equation}
Somit gilt für die potentielle Energie:
\begin{equation}
\langle V \rangle = \frac{m\omega^2\langle x^2\rangle}{2} = \frac{1}{4} \omega \hbar
\end{equation}

Die potentielle Energie $V=Dx^2/2$ ist eine homogene Funktion vom Grad $k=2$, somit liefert der Virialsatz:
\begin{equation}
\langle T \rangle = \frac{k}{2}\langle V \rangle = \langle V \rangle
\end{equation}
Dieses Ergebnis stimmt mit der obigen Rechnung überein.
\item Für die Ortsunschärfe gilt
\begin{equation}
\Delta x = \sqrt{\langle x^2 \rangle - \langle x \rangle^2} = \sqrt{ \frac{\hbar}{2m\omega}},
\end{equation}
während für die Impulsunschärfe
\begin{equation}
\Delta p = \sqrt{\langle p^2 \rangle - \langle p \rangle^2} = \sqrt{\frac{1}{2}m\omega\hbar}
\end{equation}
gilt. Dabei wurde $\langle x \rangle = \langle p \rangle = 0$ verwendet.
Es folgt:
\begin{equation}
\Delta x \Delta p = \frac{\hbar}{2}
\end{equation}
\end{enumerate}
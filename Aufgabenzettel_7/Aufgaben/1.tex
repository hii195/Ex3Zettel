\section*{\nr.1 \titone (25 Punkte)}
\begin{enumerate}[(a)]
\item Ohne anliegende Spannung erhält man eine Tunnelwahrscheinlichkeit von
\begin{equation}
  P=\exp\left(-2d \frac{\sqrt{2mq\phi}}{\hbar}\right)=1.07\cdot 10^{-20}
\end{equation}
Für die Spannungen 5,10 und 15V erhält man
\begin{align}
  P(5V)&=2.23\cdot 10^{-11}\\
  P(10V)&=4.72\cdot 10^{-6}\\
  P(15V)&=2.81\cdot 10^{-4}
\end{align}
\item Die \glqq Anklopfrate\grqq $f$ ergibt sich aus
\begin{equation}
  f=\frac{v_{R}}{2s}
\end{equation}
wobei $s$ die Dicke des Gates bezeichnet. Damit folgt
\begin{equation}
  f=\sqrt{\frac{kT}{8s^2\pi m}}
\end{equation}
was mit der Annahme, dass das Gate eine (Raum-)Temperatur von $T=293K$ hat, $f=3.32\cdot 10^{13}\frac{1}{\mathrm{s}}$ ergibt.
Damit erhält man das Zerfallsgesetz
\begin{equation}
  N(t)=N_0\exp\left(-Pft\right)
\end{equation}
und mit der Bedingung 
\begin{equation}
  \frac{N(t)}{N(0)}=0.9 \,\,\mathrm{bzw.} \,\, \frac{N(t)}{N(0)}=0.5 
\end{equation}
erhält man eine Zeit $t_{0.9}=2.98\cdot 10^{5}$s und $t_{0.5}=1.96\cdot 10^{6}$s
\item Nimmt man an, dass für den Löschvorgang eine Spannung von $5$V angelegt wird, so erhält man 
\begin{equation}
  t=3.11\cdot 10^{-3}s
\end{equation}
\end{enumerate}
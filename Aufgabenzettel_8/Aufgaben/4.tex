\section*{\nr.4 \titfour (25 Punkte)}
\begin{enumerate}[(a)]
\item Mit dem Rosinenkuchenmodell würde man eine gaußförmige Winkelabhängigkeit bei der Streuung erwarten, die deutlich schneller abfällt als die gemessene und durch das Rutherfordsche Modell vorhergesagte $\frac{1}{\sin(\theta/2)^4}$ Abhängigkeit. Dieser Unterschied macht sich besonders bei großen Streuuwinkeln bemerkbar. Rutherford folgerte aus dieser Abhängigkeit, dass die meiste Masse sich im potitiv geladenen Kern befindet, die einen gegenüber des Atomvolumen kleines Volumen einnimmt, um den sich die negativ geladenen Elektronen kreisen.
\item Da $v_i<\frac{c}{10}$ wird im folgenden nicht relativistisch gerechnet.\\
Im Unendlichen hat das $\alpha$-Teilchen die Energie
\begin{equation}
  E_{kin}=\frac{1}{2}m_{\alpha}v_i^2.
\end{equation}
Im Umkehrpunkt im Abstand $r$ vom Kern muss diese komplett in Form von potentieller Energie gespeichert sein
\begin{equation}
  E_{pot}=\frac{2e\cdot79e}{4\pi\epsilon_0}\frac{1}{r}
\end{equation}
es folgt damit nach gleichsetzen
\begin{equation}
  r=\frac{79e^2}{\pi\epsilon_0m_{\alpha}v_i^2}=2.74\cdot 10^{-14}\mathrm{m}
\end{equation}
\item Im Umkehrpunkt wirkt die Coulombkraft
\begin{equation}
  F(r)=\frac{2e\cdot 79e}{4\pi\epsilon_0 r^2}=m_{\alpha}a
\end{equation}
womit nach Umformen
\begin{equation}
  a=7.31\cdot10^{27}\frac{\mathrm{m}}{\mathrm{s}^2}
\end{equation}
folgt.
\end{enumerate}
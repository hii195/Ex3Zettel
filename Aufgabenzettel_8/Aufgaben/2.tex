\section*{\nr.2 \tittwo (25 Punkte)}
\begin{enumerate}[(a)]
\item Aus der Form $\hat{L^2}\psi = 2IE \psi$ der Schrödingergleichung folgt, dass es ausreicht zu zeigen, dass $\psi=A\exp(-2i\phi)\sin^2{\theta}$ eine Eigenfunktion des Operators des Drehimpulsquadrates ist:
\begin{align}
\hat{L^2}\psi &= -\hbar^2 \left[ \frac{1}{\sin{\theta}}\frac{\partial}{\partial\theta}\left(\frac{\partial}{\partial\theta}\psi \right) + \frac{1}{\sin^2{\theta}}\frac{\partial^2}{\partial \phi^2}\psi\right] \\
&= -\hbar^2 \left[ \frac{1}{\sin{\theta}}\frac{\partial}{\partial\theta}\left(\frac{\partial}{\partial\theta}\left(A\exp(-2i\phi)\sin^2{\theta}\right) \right) + \frac{1}{\sin^2{\theta}}\frac{\partial^2}{\partial \phi^2}\left(A\exp(-2i\phi)\sin^2{\theta}\right) \right] \\
&= -\hbar^2 \left[ 2A\exp(-2i\phi)\left(2-3\sin^2{\theta}\right)-4A\exp(-2i\phi)\right] \\
&= 6 \hbar^2 \psi 
\end{align}
$\psi$ ist also eine Eigenfunktion des Operators mit Eigenwert $6\hbar^2$. Die Drehimpulsquantenzahl ist somit $l=2$, denn $6=2\cdot(2+1)$. 
\item Für das Quadrat des Drehimpulses gilt:
\begin{equation}
\langle L^{2} \rangle = \hbar^2 l(l+1) = 6\hbar^2 
\end{equation}
\item Berechnen wir:
\begin{equation}
\hat{L_z}\psi = -i\hbar \frac{\partial}{\partial \phi} \psi = -2\hbar \psi
\end{equation}
Der Eigenwert dieser Gleichung ist die gesuchte z-Komponente des Drehimpulses: $\langle L_z\rangle = -2\hbar$. Nebenbei beträgt die magnetische Quantenzahl somit $l=-2$.
\end{enumerate}
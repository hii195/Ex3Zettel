\section*{\nr.1 \titone (25 Punkte)}
\begin{enumerate}[(a)]
\item Im Grundzustand $n=0$ ist das Hermite'sche Polynom durch $H_0(z)=1$ gegeben. Es folgt:
\begin{align}
\psi_0 (x) &= \frac{\sqrt{\alpha}}{\sqrt[4]{\pi}} \exp\left( -\frac{\alpha^2x^2}{2}\right) \\
\left|\psi_0 (x) \right|^2 &= \frac{\alpha}{\sqrt{\pi}} \exp\left( -\alpha^2x^2\right) \\
E_0 &= \frac{\hbar\omega}{2}
\end{align}
Lässt sich der harmonische Oszillator mit der Federkonstante $D=m\omega^2$ charakterisieren, so folgen aus Gleichsetzen von $E_0$ mit der potentiellen Energie $V=Dx^2/2$ die Umkehrpunkte:
\begin{equation}
x_{1/2} = \pm \sqrt{\frac{\hbar\omega}{D}} = \pm \sqrt{\frac{\hbar}{m\omega}}
\end{equation}
Die Wahrscheinlichkeit, dass sich ein Teilchen im klassisch verbotenen Bereich aufhält, errechnet sich somit zu:
\begin{equation}
P(\text{außen}) = 1- \int_{x_1}^{x_2}{\left|\psi_0 (x) \right|^2\,\mathrm{d}x} = 1- \int_{-1}^{1}{\frac{1}{\sqrt{\pi}} \exp(-x^2)\,\mathrm{d}x} \approx \num{0.157}
\end{equation}
\item Die Zeitentwicklung der Wellenfunktion ist durch 
\begin{equation}
\psi_n(x,t) = \psi_n(x) \exp(-i\omega t)
\end{equation}
gegeben, das heißt, sie oszilliert mit der Kreisfrequenz $\omega$.
\item Beobachtet man die Einzelzustände (ohne Superposition), so ist die Aufenthaltswahrscheinlichkeitsdichte zeitlich konstant und "`hügelig"', wobei die Anzahl der Hügel mit zunehmendem \texttt{nmax} ebenfalls zunimmt. Weiter wird die Verteilung bei höheren Zuständen zunehmend breiter, da mit höherer kinetischer Energie die Umkehrpunkte weiter von der von der Ruhelage entfernt sind.

Bei einer (gleichverteilten) Überlagerung aller Zustände bis zu einem \texttt{nmax} nun nicht mehr zeitlich konstant, sondern schwingt in Form eines Wellenpakets zwischen den Umkehrpunkten. 

\item Bei einer poissonverteilten Superposition der Zustände entsteht mit zunehmendem \texttt{nmax} eine saubere Gaußkurve, die periodisch zwischen den Umkehrpunkten oszilliert. Dieses Grenzverhalten ist bei hinreichend hohen Zuständen unabhängig von \texttt{nmax}.
\end{enumerate}
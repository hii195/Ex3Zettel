\section*{\nr.3 \titthree (25 Punkte)}
\begin{enumerate}[(a)]
\item Es gilt nach PTP3
\begin{align}
  Y_1^0\left(\theta,\phi\right)&=\sqrt{\frac{3}{4\pi}}\cos\theta\\
  Y_1^{\pm1}\left(\theta,\phi\right)&=\mp\sqrt{\frac{3}{8\pi}}\sin\theta\exp\left(\pm i\phi\right)\\
  &=\mp\sqrt{\frac{3}{8\pi}}\sin\theta\left(\cos(\pm\phi)+i\sin(\pm\phi)\right).
\end{align}
Durch einfache trigonometrische Überlegungen erhält man außerdem
\begin{align}
  \cos\theta&=\frac{z}{r}\\
  \sin\theta&=\frac{\sqrt{x^2+y^2}}{r}\\
  \sin(\pm\phi)&=\frac{\pm y}{\sqrt{x^2+y^2}}\\
  \cos(\pm\phi)&=\frac{x}{\sqrt{x^2+y^2}}.
\end{align}
Damit folgen nach einsetzen und kürzen direkt die geforderten Formeln
\begin{align}
  Y_1^0(\theta,\phi)&=\sqrt{\frac{3}{4\pi}}\frac{z}{r}\\
  Y_1^{\pm 1}(\theta,\phi)&=\mp \sqrt{\frac{3}{8\pi}}\frac{x\pm iy}{r}
\end{align}

\end{enumerate}
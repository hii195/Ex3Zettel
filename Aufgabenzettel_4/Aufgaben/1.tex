\section*{\nr.1 \titone (25 Punkte)}
\begin{enumerate}[(a)]
\item Die Ringe kommen durch die polykristalline Struktur zustande. Durch diese kann das System in erster Näherung als Rotationssymmetrisch angenommen werden, es gibt keine bevorzugte Richtung für die Lage der Kristallite, weshalb auch das Beugungsmuster diese Symmetrie widerspigeln muss.
\item Es gilt die de-Broglie Beziehung
\begin{equation}
  \lambda = \frac{h}{p}
\end{equation}
mit $E=\frac{p^2}{2m}$ und $E=V_ae$ folgt
\begin{equation}
  \lambda = \frac{h}{\sqrt{2m V_a e}}
\end{equation}
\item Es gilt die Bragg-Gleichung (für $n=1$)
\begin{equation}
  \lambda=2d\sin{\theta}
\end{equation}
mit $\theta = \pi-\arctan \frac{D}{a}$, $\sin(\pi-\arctan x)=\frac{x}{\sqrt{1+x^2}}$ und obiger Gleichung folgt 
\begin{equation}
  \frac{h}{2mV_a e}=2d \frac{D}{a}\frac{1}{\sqrt{1+\left(\frac{D}{a}\right)^2}}
\end{equation}
woraus mit einigen Umformungen 
\begin{equation}
  D=\sqrt{\frac{h^2a^2}{8mV_a qd^2-h^2}}
\end{equation}
folgt. Da $h^2\ll8mV_a qd^2$ (siehe unten) kann man dies zu
\begin{equation}
   D=\sqrt{\frac{h^2a^2}{8m qd^2}}\frac{1}{\sqrt{V_a}}
\label{eq:D}
\end{equation}
vereinfachen.
\item Das Diagramm ist mit \vref{fig:elektronenbeugung} gegeben.
\begin{figure}[htbp]
\centering
% GNUPLOT: LaTeX picture with Postscript
\begingroup
  \makeatletter
  \providecommand\color[2][]{%
    \GenericError{(gnuplot) \space\space\space\@spaces}{%
      Package color not loaded in conjunction with
      terminal option `colourtext'%
    }{See the gnuplot documentation for explanation.%
    }{Either use 'blacktext' in gnuplot or load the package
      color.sty in LaTeX.}%
    \renewcommand\color[2][]{}%
  }%
  \providecommand\includegraphics[2][]{%
    \GenericError{(gnuplot) \space\space\space\@spaces}{%
      Package graphicx or graphics not loaded%
    }{See the gnuplot documentation for explanation.%
    }{The gnuplot epslatex terminal needs graphicx.sty or graphics.sty.}%
    \renewcommand\includegraphics[2][]{}%
  }%
  \providecommand\rotatebox[2]{#2}%
  \@ifundefined{ifGPcolor}{%
    \newif\ifGPcolor
    \GPcolorfalse
  }{}%
  \@ifundefined{ifGPblacktext}{%
    \newif\ifGPblacktext
    \GPblacktexttrue
  }{}%
  % define a \g@addto@macro without @ in the name:
  \let\gplgaddtomacro\g@addto@macro
  % define empty templates for all commands taking text:
  \gdef\gplbacktext{}%
  \gdef\gplfronttext{}%
  \makeatother
  \ifGPblacktext
    % no textcolor at all
    \def\colorrgb#1{}%
    \def\colorgray#1{}%
  \else
    % gray or color?
    \ifGPcolor
      \def\colorrgb#1{\color[rgb]{#1}}%
      \def\colorgray#1{\color[gray]{#1}}%
      \expandafter\def\csname LTw\endcsname{\color{white}}%
      \expandafter\def\csname LTb\endcsname{\color{black}}%
      \expandafter\def\csname LTa\endcsname{\color{black}}%
      \expandafter\def\csname LT0\endcsname{\color[rgb]{1,0,0}}%
      \expandafter\def\csname LT1\endcsname{\color[rgb]{0,1,0}}%
      \expandafter\def\csname LT2\endcsname{\color[rgb]{0,0,1}}%
      \expandafter\def\csname LT3\endcsname{\color[rgb]{1,0,1}}%
      \expandafter\def\csname LT4\endcsname{\color[rgb]{0,1,1}}%
      \expandafter\def\csname LT5\endcsname{\color[rgb]{1,1,0}}%
      \expandafter\def\csname LT6\endcsname{\color[rgb]{0,0,0}}%
      \expandafter\def\csname LT7\endcsname{\color[rgb]{1,0.3,0}}%
      \expandafter\def\csname LT8\endcsname{\color[rgb]{0.5,0.5,0.5}}%
    \else
      % gray
      \def\colorrgb#1{\color{black}}%
      \def\colorgray#1{\color[gray]{#1}}%
      \expandafter\def\csname LTw\endcsname{\color{white}}%
      \expandafter\def\csname LTb\endcsname{\color{black}}%
      \expandafter\def\csname LTa\endcsname{\color{black}}%
      \expandafter\def\csname LT0\endcsname{\color{black}}%
      \expandafter\def\csname LT1\endcsname{\color{black}}%
      \expandafter\def\csname LT2\endcsname{\color{black}}%
      \expandafter\def\csname LT3\endcsname{\color{black}}%
      \expandafter\def\csname LT4\endcsname{\color{black}}%
      \expandafter\def\csname LT5\endcsname{\color{black}}%
      \expandafter\def\csname LT6\endcsname{\color{black}}%
      \expandafter\def\csname LT7\endcsname{\color{black}}%
      \expandafter\def\csname LT8\endcsname{\color{black}}%
    \fi
  \fi
    \setlength{\unitlength}{0.0500bp}%
    \ifx\gptboxheight\undefined%
      \newlength{\gptboxheight}%
      \newlength{\gptboxwidth}%
      \newsavebox{\gptboxtext}%
    \fi%
    \setlength{\fboxrule}{0.5pt}%
    \setlength{\fboxsep}{1pt}%
\begin{picture}(7200.00,5040.00)%
    \gplgaddtomacro\gplbacktext{%
      \csname LTb\endcsname%
      \put(1078,704){\makebox(0,0)[r]{\strut{}$0.04$}}%
      \put(1078,1444){\makebox(0,0)[r]{\strut{}$0.042$}}%
      \put(1078,2184){\makebox(0,0)[r]{\strut{}$0.044$}}%
      \put(1078,2925){\makebox(0,0)[r]{\strut{}$0.046$}}%
      \put(1078,3665){\makebox(0,0)[r]{\strut{}$0.048$}}%
      \put(1078,4405){\makebox(0,0)[r]{\strut{}$0.05$}}%
      \put(1210,484){\makebox(0,0){\strut{}$0.014$}}%
      \put(2329,484){\makebox(0,0){\strut{}$0.015$}}%
      \put(3447,484){\makebox(0,0){\strut{}$0.016$}}%
      \put(4566,484){\makebox(0,0){\strut{}$0.017$}}%
      \put(5684,484){\makebox(0,0){\strut{}$0.018$}}%
      \put(6803,484){\makebox(0,0){\strut{}$0.019$}}%
    }%
    \gplgaddtomacro\gplfronttext{%
      \csname LTb\endcsname%
      \put(176,2739){\rotatebox{-270}{\makebox(0,0){\strut{}$D[\mathrm{m}]$}}}%
      \put(4006,154){\makebox(0,0){\strut{}$V_a^{-1/2}[\mathrm{V}^{-1/2}]$}}%
      \csname LTb\endcsname%
      \put(5816,4602){\makebox(0,0)[r]{\strut{}data}}%
      \csname LTb\endcsname%
      \put(5816,4382){\makebox(0,0)[r]{\strut{}Fit}}%
    }%
    \gplbacktext
    \put(0,0){\includegraphics{elektronenbeugung}}%
    \gplfronttext
  \end{picture}%
\endgroup

\caption{Ringdurchmesser in Abhängigkeit der inversen Wurzel der Beschleunigungsspannung}
\label{fig:elektronenbeugung}
\end{figure}
Mit Mathematika wurde ein Linearer least-square-Fit der Form $f(x)=cx$ zu den Datenpunkten durchgeführt. Dafür erhält man
\begin{equation}
  f(x)= 2.253 x.
\end{equation}
Die kann man nun mit dem Vorfaktor in \vref{eq:D} gleichsetzen womit man
\begin{equation}
  \sqrt{\frac{h^2a^2}{8m qd^2}} = c 
\end{equation}
erhält, was man zu 
\begin{equation}
  d=\sqrt{\frac{h^2a^2}{8m qc^2}}=2.97141\cdot 10^{-11} m
\end{equation}
umschreiben kann. Man sieht nun insbesondere auch, dass die Annahme $h^2\ll8mV_a qd^2$ vertretbar ist, da $h^2=4.39\cdot10^{-67} \mathrm{Js}$ und $8mV_a qd^2=3.09\cdot10^{-66} \mathrm{Js}$.
\end{enumerate}


\section*{\nr.2 \tittwo (25 Punkte)}
\begin{enumerate}[(a)]
\item Es gilt die de Broglie Beziehung
\begin{equation}
  \lambda=\frac{h}{p}.
\end{equation}
Damit sich die Wellenlänge halbiert muss also gelten, dass
\begin{equation}
  \frac{1}{2}=\frac{\lambda_1}{\lambda_2}=\frac{p_2}{p_1}=\frac{v_2}{v_1}
\end{equation}
mit $v_2=gt$ folgt
\begin{equation}
  t=\frac{v_1}{2g}=0.05\mathrm{s}
\end{equation}
\item Aus der gegebenen Impulsunschärfe zum Zeitpunkt $t=0$ folgt zunächst 
\begin{equation}
  \Delta y_0=\frac{\hbar}{\Delta p_{y,0}}=1.05\cdot 10^{-7}\mathrm{m}
\end{equation}
Für die Unschärfe des Ortes zur Zeit $t$ muss
\begin{equation}
  \Delta y(t)=\Delta y_0+\Delta v_g t
\end{equation}
gelten, wobei für die Unschärfe der Gruppengeschwindigkeit des Wellenpaketes $\Delta v_g$ 
\begin{equation}
  \Delta v_g = \frac{\Delta p_y}{m}= \frac{\hbar}{m\Delta y_0}
\end{equation}
gilt.
Mit der Bedingung $\Delta y(t)=2\Delta y_0$ folgt
\begin{equation}
\frac{\hbar}{m\Delta y_0}t=\Delta y_0\implies t=\frac{m\Delta y_0^2}{\hbar}=\frac{m\hbar}{\Delta p_{y,0}^2}=2.74\cdot10^{15}\mathrm{s}
\end{equation}
\end{enumerate}
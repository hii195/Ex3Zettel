\section*{\nr.4 \titfour (25 Punkte)}
\begin{enumerate}[(a)]
\item In die Formel für die relativistische Gesamtenergie
\begin{equation}
E = \sqrt{p^2c^2+m_e^2c^4}
\end{equation}
werden die Beziehungen $E=\hbar \omega$ und $p= \hbar k$ eingesetzt und nach $\omega$ aufgelöst. Dies führt direkt zur Dispersionsrelation:
\begin{equation}
\omega(k) = \sqrt{k^2c^2+\frac{m_e^2c^4}{\hbar^2}}
\end{equation}
\item Die Phasengeschwindigkeit errechnet sich zu:
\begin{equation}
v_p = \frac{\omega}{k} = \sqrt{c^2+\frac{m_e^2c^4}{k^2\hbar^2}}
\end{equation}
Dabei fällt auf, dass sie größer als die Lichtgeschwindigkeit ist. Dies steht jedoch nicht im Widerspruch zur speziellen Relativitätstheorie, da man \emph{nur} mit der Gruppengeschwindigkeit Informationen übertragen kann, jedoch nicht mit der Phasengeschwindigkeit. Die Gruppengeschwindigkeit beträgt:
\begin{equation}
v_g = \frac{\mathrm{d}\omega}{\mathrm{d}k} = \frac{c^2 k}{\sqrt{k^2c^2+\frac{m_e^2c^4}{\hbar^2}}}
\end{equation}
Das Produkt dieser beiden Geschwindigkeiten beträgt
\begin{equation}
v_p\cdot v_g = \frac{1}{k}\sqrt{k^2c^2+\frac{m_e^2c^4}{\hbar^2}} \frac{c^2 k}{\sqrt{k^2c^2+\frac{m_e^2c^4}{\hbar^2}}} = c^2,
\end{equation}
ist also unabhängig von $k$ konstant.
\item Da die Phasengeschwindigkeit größer als die Lichtgeschwindigkeit ist und das Produkt beider Geschwindigkeiten $c^2$ beträgt, ist die Gruppengeschwindigkeit unterhalb der Lichtgeschwindigkeit und Einstein lächelt zufrieden im Grab.
\end{enumerate}
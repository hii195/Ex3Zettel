\section*{\nr.1 \titone (25 Punkte)}
\begin{enumerate}[(a)]
\item Es gilt die Rydbergformel
\begin{equation}
  \frac{1}{\lambda}=R_H\left(\frac{1}{n_j^2}-\frac{1}{n_i^2}\right)
\end{equation}
mit $n_i>n_j$. Mit $n_i=2$ und $n_j=1$ folgt
\begin{equation}
  \lambda = \frac{4}{3R_H}=1.22\cdot10^{-7}m
\end{equation}
\item Es muss Impulserhaltung gelten, das Atom muss also nach der Absorption den Impuls des Photons haben.
\begin{equation}
  p_H = p_{ph}=\frac{h}{\lambda}
\end{equation}
Unter der Annahme, dass das Atom vor der Absorption in Ruhe war hat es anschließend die Energie
\begin{equation}
  E=\frac{p^2}{2m}=\frac{h^2}{2m\lambda^2}=8.95\cdot 10^-27\mathrm{J}
\end{equation}
und die de-Broglie-Wellenlänge
\begin{equation}
  \lambda_H=\lambda_{ph}=1.22\cdot 10^{-7}m
\end{equation}

Es sind in der Abbildung die ersten 4 Linien der Balmerserie zu sehen, dies sind die Übergänge vom 3.,4.,5. und 6. Energieniveau ins 2. Mit der Rydberg-Formel erhält man als entsprechende Wellenlängen
\begin{align}
  \lambda_{3,2}&=656.3\mathrm{nm}\\
  \lambda_{4,2}&=486.1\mathrm{nm}\\
  \lambda_{5,2}&=434.1\mathrm{nm}\\
  \lambda_{6,2}&=410.1\mathrm{nm}
\end{align}
wobei die rote Linie der Übergang von 3 nach 2 ist und die Linien links daneben in entsprechender Reihenfolge den anderen Übergängen entsprechen.  

\end{enumerate}
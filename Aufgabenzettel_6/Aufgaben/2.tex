\section*{\nr.2 \tittwo (25 Punkte)}
\begin{enumerate}[(a)]
\item Im Bereich I kann sich das Teilchen nicht aufhalten, es muss also
\begin{equation}
  \psi_1=0
\end{equation}
gelten für $x<0$. Im Bereich II erhält man eine Überlagerung einer rechts- und einer linksläufigen Welle
\begin{equation}
  \psi_2=A \exp\left(-ik_1x\right)+B\exp\left(ik_1x\right).
\end{equation}
Im Bereich III gilt
\begin{equation}
  \psi_3=C \exp\left(-\alpha x \right)
\end{equation}
\item Die Wellenfunktion muss an allen Stellen stetig sein, es gilt also für die Übergänge
\begin{equation}
  \psi_1(0)=\psi_2(0)=0
\end{equation}
und 
\begin{equation}
  \psi_2(L)=\psi_3(L).
\end{equation}
Außerdem muss die Wellenfunktion an allen Stellen mit endlichem Potential differenzierbar sein, woraus
\begin{equation}
  \left(\frac{\mathrm{d}}{\mathrm{d}x}\psi_2\right)\left(L\right)=\left(\frac{\mathrm{d}}{\mathrm{d}x}\psi_3\right)\left(L\right)
\end{equation}
folgt.
\item Damit die Wellenfunktion normierbar bleibt, muss sie im Unendlichen verschwinden. Es gilt also
\begin{equation}
  \lim_{x \to \pm \infty}\psi(x)=0.
\end{equation}
\end{enumerate}
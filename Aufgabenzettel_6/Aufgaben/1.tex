\section*{\nr.1 \titone (25 Punkte)}
Das Interferenzmuster $I(y)$ ergibt sich über Fouriertransformation der Spaltlandschaft $L$
\begin{equation}
  I(y)=|\int_{-\infty}^{\infty} \mathrm{d}x \chi_L \frac{1}{\sqrt{2 \pi}}\exp\left(ixy\right)|^2
\end{equation}
Wobei $\chi_L$ die charakteristische Funktion der Spaltlandschaft ist 
\begin{equation}
   \chi_L(x) =
   \begin{cases}
     1 & \text{f"ur } 0 \leq x \leq d_1 \\
     1 & \text{f"ur } a + d_1 \leq x \leq a + d_1 + d_2 \\
     0  & \text{sonst}
   \end{cases}
\end{equation}
Durch die Wahl von $d_2 = \frac{d_1}{25}$ ist die Bedingung dass durch den einen Spalt $25$ mal mehr Elektronen kommen sollen erfüllt, da
\begin{equation}
  \int_{-\infty}^{\infty} \mathrm{d}y |\int_{0}^{d_1} \mathrm{d}x \frac{1}{\sqrt{2 \pi}}\exp\left(ixy\right)|^2 =d_1
\end{equation}
und 
\begin{equation}
  \int_{-\infty}^{\infty} \mathrm{d}y |\int_{a+d_1}^{a+d_1+\frac{d_1}{25}} \mathrm{d}x \frac{1}{\sqrt{2 \pi}}\exp\left(ixy\right)|^2 =\frac{d_1}{25}
\end{equation}
Die Ausführung des Fourierintegrals wurde mit Mathematika durchgeführt und numerisch nach dem ersten Minimum gesucht. Für $d_1=25$ und $a=100000$ ergibt sich damit ein Verhältnis der Wahrscheinlichkeit, dass sich das Elektron in einem Interferenzmaximum landet, zu der, dass es in einem Interferenzminimum landet von
\begin{equation}
  \frac{p_{max}}{p_{min}}=1.17.
\end{equation}
Alternativ kann man auch argumentieren, dass die Amplitude der Wellenfunktion durch den einen Spalt $25$ mal größer ist als die Amplitude der Wellenfunktion, die durch den anderen Spalt geht. Bei konstruktiver Interferenz erhält man also eine Amplitude von $26$ (in beliebigen Einheiten), bei destruktiver Interferenz eine Amplitude von $24$. Damit folgt für die Aufenthaltswahrscheinlichkeit des Teilchens im Maximum im Vergleich zum Aufenthalt im Minimum
\begin{equation}
  \frac{p_{max}}{p_{min}}=\frac{26^2}{24^2}=1.17
\end{equation}
was dem numerischen Ergebnis entspricht.


